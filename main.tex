\documentclass{article}
\usepackage[utf8]{inputenc}
\usepackage[spanish]{babel}
\usepackage{listings}
\usepackage{graphicx}
\graphicspath{ {images/} }
\usepackage{cite}

\begin{document}

\begin{titlepage}
    \begin{center}
        \vspace*{1cm}
            
        \Huge
        \textbf{Callistenia}
            
        \vspace{0.5cm}
        \LARGE
        
            
        \vspace{1.5cm}
            
        \textbf{Jeisson Arley Alvarez Giraldo}
            
        \vfill
            
        \vspace{0.8cm}
            
        \Large
        Despartamento de Ingeniería en Telecomunicaciones\\
        Universidad de Antioquia\\
        Medellín\\
        Marzo de 2021
            
    \end{center}
\end{titlepage}

\tableofcontents
\newpage
\section{Reglas y datos preliminares}\label{intro}
Objetos necesarios:

•	1 Hoja de papel

•	2 tarjetas tipo cedula con pesos similares

•	1 superficie plana lo suficientemente grande para que el papel quede totalmente apoyado


Posición inicial:

Se hará referencia como posición inicial del ejercicio cuando las cedulas se encuentren sobre la superficie plana y la hoja de papel se encuentre extendida sobre ambas y totalmente encima de la superficie


Conocimientos necesarios:

•	Tener conocimientos sobres sistemas de ejes coordenados del plano cartesiano

•	Saber el nombre de los dedos de la mano

•	Conocer la diferencia entre horizontal y vertical


Reglas:

•	Solo debe seguir las instrucciones descritas a continuación como las interprete y no realizar ninguna pregunta

•	Solo debe usar su mano dominante para seguir las instrucciones






\newpage\section{Instrucciones paso a paso} \label{contenido}









Pasos a seguir:

Siguiendo las reglas previamente descritas se procede a:

*	Primero se deben ubicar los objetos en la posición inicial como se referencia anteriormente.

*	Se levanta la hoja de papel de la superficie plana ubicándola a un lado de esta de forma que las tarjetas queden expuestas y se puedan manipular

*	Se recogen las tarjetas y se colocan a un lado

*	Se ubica la hoja de papel en un lugar de la superficie plana donde quede perfectamente extendida y apoyada

*	Se colocan las tarjetas acostadas de forma horizontal sobre el centro de la hoja de papel alineadas perfectamente una sobre la otra

*	Como las tarjetas son un rectángulo se les van asignar un sistema de coordenadas basadas en el plano cartesiano, “Y” siendo la parte más larga y “X” las angosta teniendo en cuenta la posición en la que quedaron las tarjetas en el paso anterior

*	Se procede a tomar las tarjetas por la parte superior del eje “Y” usando los dedos índice y del medio, usando el dedo pulgar para tomar las tarjetas por su lateral más próximo, el dedo anular para tomar el lateral opuesto y todos los dedos al mismo tiempo

*	Levantar las tarjetas sosteniéndolas con los dedos como se describe anteriormente

*	Apoyar las  tarjetas con su eje “Y” en la parte inferior, es decir al lado opuesto al cual tiene sus dedos sobre el centro de la hoja de papel

*	 Con la uña del dedo pulgar separar un poco las tarjetas sin soltar los otros dedos y sin dejar de apoyarla sobre la hoja de papel

*	Inclinar todo el apoyo sobre la tarjeta que se encuentre en posición opuesta a su mano

*	Sin soltar los dedos índice y del medio use los dedos pulgar y anular para ir separando lentamente las tarjetas desde la base sin separarlas en la parte superior del eje “Y” hasta que encuentre un punto de equilibrio entre las dos tarjetas y se forme una pirámide entre ellas, sienta libertad de mover los dedos en las superficies de las tarjetas de la forma en que más comodidad le genere

*	Una vez las tarjetas se encuentren en perfecto equilibrio obteniendo la forma piramidal suelte las tarjetas





\end{document}
